%%%%%%%%%% default packages %%%%%%%%%%

%%%%% basic functions %%%%%

\usepackage[margin=2cm]{geometry} % 邊界、紙張、頁眉、頁腳、頁框,設定時可用 \geometry{<op1>, <op2>, ...}
\usepackage[onehalfspacing]{setspace}
\usepackage{comment}

%%%%% text fonts %%%%%

\usepackage{lmodern}
\usepackage{fontspec}
\usepackage[fallback]{xeCJK}

\XeTeXlinebreaklocale "zh"
\XeTeXlinebreakskip = 0pt plus 1pt

\InputIfFileExists{/home/prairie2022/texmf/fonts/font\_config.tex}{}{ % Overleaf
    \setCJKmainfont[AutoFakeSlant=.2]{Noto Serif CJK TC}
    \setCJKsansfont[AutoFakeSlant=.2]{Noto Sans CJK TC}
    \setCJKmonofont{Noto Sans Mono CJK TC}
}

%%%%% math %%%%%

\usepackage{bm} % \bm{粗體字}
\usepackage{mathrsfs} % \mathscr{花體字}
\usepackage{amsmath,amsthm,amssymb}
\usepackage{yhmath} % 優化排版
\usepackage{mathtools} % 額外排版功能
% \DeclarePairedDelimiter{\norm}{\lVert}{\rVert}
\usepackage{esint} % 積分符號
\usepackage{esdiff} % \diff(p)(*)
\usepackage{cancel} % \cancel{刪除線} \bcancel{反刪除線} \xcancel{交叉刪除線}
\usepackage{centernot} % \centernot{}
% \usepackage{siunitx} % \si{單位} \SI{數值}{單位}

%%%%% other items %%%%%

\usepackage{xcolor} % 顏色
\usepackage[inline]{enumitem} % [label= , itemjoin= ]
\usepackage{fancyhdr} % 頁首頁尾
\usepackage{graphicx} % 插入圖片 \includegraphics[]{}
\usepackage[normalem]{ulem} % 下劃線
\usepackage{titling} % \theauthor
\usepackage{xparse} % \DeclareDocumentCommand{}{}{}

% chart
\usepackage{tabularx} % 表格
\usepackage{multirow} % \multirow{num_rows}{*=自動寬度}{content}
% 類似功能有 \multicolumn{num_cols}{c/l/r}{content}

% link
\usepackage{xurl} % break long url
\usepackage[unicode, colorlinks]{hyperref} % 超連結

% draw
\usepackage{tikz} % \usetikzlibrary{}
\usepackage{cleveref} % \label{}, \ref{}, \cref{}

% code
\usepackage[newfloat, cache=false]{minted}
% \usepackage{algorithm}
% \usepackage[noend]{algpseudocode}

%%%%%%%%%% custom commands %%%%%%%%%%

%%%%% theorem %%%%%

\theoremstyle{definition} % 預設為 plain,內文為斜體
\newtheorem{definition}{Definition}
\newtheorem{claim}{Claim}
\newtheorem{lemma}{Lemma}
\newtheorem*{remark}{Remark}
\newtheorem*{recall}{Recall}

%%%%% general math abbreviation %%%%%

\NewDocumentCommand{\OP}{u{ }}{\operatorname{#1}}
\let~\OP

\def\N{\mathbb N}
\def\Z{\mathbb Z}
\def\Q{\mathbb Q}
\def\R{\mathbb R}
\def\C{\mathbb C}
\def\eps{\varepsilon}

%%%%%% paired delimiter,加 * 會自動調整大小 %%%%%

\DeclarePairedDelimiter{\abs}{\lvert}{\rvert}
\DeclarePairedDelimiter{\ceil}{\lceil}{\rceil}
\DeclarePairedDelimiter{\floor}{\lfloor}{\rfloor}
\DeclarePairedDelimiter{\ags}{\langle}{\rangle}
\DeclarePairedDelimiter{\norm}{\lVert}{\rVert}

\makeatletter % https://github.com/fhvirus/FHW/blob/main/fhw.cls
\newcommand{\@renewmid}{\renewcommand{\mid}{%
	\mathclose{}%
	\mathchoice{\;}{\;}{\,}{\,}%
	\delimsize\vert%
	\mathchoice{\;}{\;}{\,}{\,}%
	\mathopen{}%
}}

\DeclarePairedDelimiterX{\set}[1]{\lbrace}{\rbrace}{\@renewmid #1}
\makeatother

%%%%% arrow %%%%%

\NewDocumentCommand{\To}{O{\;}}{\xrightarrow{\;\; #1 \;\;}}
\NewDocumentCommand{\maps}{mO{\To}mO{\;}mm}{
    \begin{aligned}
        #1 #2[#4] & \; #3\\
        #5 \xmapsto{\;\;\phantom{#4}\;\;} & \; #6
    \end{aligned}
}
\ProvideDocumentCommand{\xtwoheadrightarrow}{O{}m}{\mathrel{\xrightarrow[#1]{#2}\hspace{-1.2em}\xrightarrow{}\hspace{0.2em}}}

%%%%% environment %%%%%

\NewDocumentCommand{\circled}{m}{\tikz[baseline=(char.base)]{\node[shape=circle,draw,inner sep=2pt] (char) {#1};}}
% [label = \protect\circled{\arabic*}]

\NewDocumentEnvironment{stm}{m}{
	\par\noindent\underline{#1}. \ignorespaces
}{\par}

%%%%% other %%%%%

\NewDocumentCommand{\rng}{O{1}mmO{}m}{#3_{#1}#4 #5 \cdots #5 #3_{#2}#4}

\def\d{\mathop{}\!\mathrm d} % https://tex.stackexchange.com/a/60546

%%%%% additional %%%%%

% \RequirePackage{} % 重複載入宏包不報錯
% \DeclareDocumentCommand{}{}{} % 是否已存在皆不報錯
% 還不會寫 command of command
